\documentclass{article}
\usepackage[utf8]{inputenc}
\usepackage{color}   % able to change the color of links
\usepackage{hyperref}
\hypersetup{
    colorlinks=true, % set true if you want colored links
    linktoc=all,     % set to all if you want both sections and subsections linked
    linkcolor=blue,  % choose some color if you want links to stand out
}

\title{Notes on \\ \textit{A Mathematics Course for Political and Social Research}}
\author{Enbang Wu}
\begin{document}

\maketitle

\section{Citation}
This is a note for \\ \textit{A Mathematics Course for Political and Social Research} by Will H.Moore and David A.Siegel.

\section{Keywords}
Calculus; Probability; Linear Algebra


\section{Commentary}
\begin{itemize}
    \item This note is created by Enbang Wu at the University of Wisconsin-madison for the UWM political science graduate math camp 22-23.
    \item This note is mainly to clarify some important concepts and make it easier for readers.
    \item The range of this note is from Chapter 1 through Chapter 13.
    \item This note is an open source project under the MIT license and welcome any contribution!
\end{itemize}

\section{Contribution}
Contribute to this note by making pull requests on [githublink]

\newpage
\tableofcontents
\newpage

\section{Chapter 1 Preliminaries Math}
In this chapter, it introduces some of the most fundamental math concepts. To better understand them, we will provide some metaphors.
\subsection{Sets:}
\begin{itemize}
    \item Countable sets: A countable set compromises discrete elements. You can think of it like there is a hole or distance among the elements, no matter how close they are. 
    \item Uncountable sets: A uncountable set compromises continues elements. In mathematical language, no matter how small a distance is given, I can always find elements that fit in.
\end{itemize}
Countable sets and uncountable sets are important because you will find that they share a similar idea with calculus and calculus-based probability. 

\section{Chapter 2 Algebra Review}
\end{document}
