\documentclass{article}
\usepackage[utf8]{inputenc}
\usepackage{color}   % able to change the color of links
\usepackage{hyperref}
\hypersetup{
    colorlinks=true, % set true if you want colored links
    linktoc=all,     % set to all if you want both sections and subsections linked
    linkcolor=blue,  % choose some color if you want links to stand out
}
\title{Notes on \\ \textit{A Mathematics Course for Political and Social Research}}
\author{Enbang Wu}
\begin{document}

\maketitle

\section{Citation}
This is a note for \\ \textit{A Mathematics Course for Political and Social Research} by Will H.Moore and David A.Siegel.

\section{Keywords}
Calculus; Probability; Linear Algebra


\section{Commentary}
\begin{itemize}
    \item This note is created by Enbang Wu at the University of Wisconsin-madison for the UWM political science graduate math camp 22-23.
    \item This note is mainly to clarify some important concepts and make it easier for readers.
    \item The range of this note is from Chapter 1 through Chapter 13. And not all of the chapters are covered, but you can contribute to the missing part.
    \item This note is an open source project under the MIT license and welcome any contribution!
\end{itemize}

\section{Contribution}
Contribute to this note by making pull requests on [githublink]

\newpage
\tableofcontents
\newpage

\section{Chapter 1 Preliminaries Math}
In this chapter, it introduces some of the most fundamental math concepts. To better understand them, we will provide some metaphors.
\subsection{Sets:}
\begin{itemize}
    \item Countable sets: A countable set compromises discrete elements. You can think of it like there is a hole or distance among the elements, no matter how close they are. 
    \item Uncountable sets: A uncountable set compromises continues elements. In mathematical language, no matter how small a distance is given, I can always find elements that fit in.
\end{itemize}
Countable sets and uncountable sets are important because you will find that they share a similar idea with calculus and calculus-based probability. 

\section{Chapter 3 Functions, Relations and Utility}
\subsection{Difference between functions and equations}
Functions and equations are both widely used in our life but they are different. A function is a \textbf{relation} that associate the input to the output. A equation is a \textbf{statement} that two expressions are equal.An function can often be written as equation, but not every equation is a function. 

\subsection{Nonlinear Functions: Exponents, Logarithms, and Radicals}
In this part, you should take extra attentions to the rules of nonlinear functions because they are different than what you have learner before. \newline For exponents:
\begin{itemize}
    \item Multiplication: \(x^m \times x^n = x^{m+n} \) 
    \item Division: \(x^m \div x^n = x^{m-n} \)
\end{itemize}

\newline For logarithms:
Logarithms can be thought as the inverse of exponents. The logarithm of \(x^m\) is \(m\log(x)\).
\newline For radicals:
Roots are those are those numbers represented by the radical symbol.
\end{document}
