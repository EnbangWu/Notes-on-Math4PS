\documentclass{article}
\usepackage[utf8]{inputenc}
\usepackage{color}   % able to change the color of links
\usepackage{hyperref}
\hypersetup{
    colorlinks=true, % set true if you want colored links
    linktoc=all,     % set to all if you want both sections and subsections linked
    linkcolor=blue,  % choose some color if you want links to stand out
}
% TODO: check the Latex Workshop documentation for Mac installtion
\title{Notes on \\ \textit{A Mathematics Course for Political and Social Research}}
\author{Enbang Wu}
\begin{document}

\maketitle

\section{Citation}
This is a note for \\ \textit{A Mathematics Course for Political and Social Research} by Will H.Moore and David A.Siegel.

\section{Keywords}
Calculus; Probability; Linear Algebra


\section{Commentary}
\begin{itemize}
    \item This note is created by Enbang Wu at the University of Wisconsin-madison for the UWM political science graduate math camp 22-23.
    \item This note is mainly to clarify some important concepts and make it easier for readers.
    \item The range of this note is from Chapter 1 through Chapter 13. And not all of the chapters are covered, but you can contribute to the missing part.
    \item This note is an open source project under the MIT license and welcome any contribution!
\end{itemize}

\section{Contribution}
Contribute to this note by making pull requests on this \hyperlink{https://github.com/EnbangWu/Notes-on-Math4PS}{Github repo}

\newpage
\tableofcontents
\newpage

\section{Preliminaries Math, Chap 1-4}
In this chapter, it introduces some of the most fundamental math concepts. To better understand them, we will provide some metaphors.
\subsection{Sets:}
\begin{itemize}
    \item Countable sets: A countable set compromises discrete elements. You can think of it like there is a hole or distance among the elements, no matter how close they are. 
    \item Uncountable sets: A uncountable set compromises continues elements. In mathematical language, no matter how small a distance is given, I can always find elements that fit in.
\end{itemize}
Countable sets and uncountable sets are important because you will find that they share a similar idea with calculus and calculus-based probability. 


\subsection{Difference between functions and equations}
Functions and equations are both widely used in our life but they are different. A function is a \textbf{relation} that associate the input to the output. A equation is a \textbf{statement} that two expressions are equal.An function can often be written as equation, but not every equation is a function. 

\subsection{Nonlinear Functions: Exponents, Logarithms, and Radicals}
In this part, you should take extra attentions to the rules of nonlinear functions because they are different than what you have learner before. \newline For exponents:
\begin{itemize}
    \item Multiplication: \(x^m \times x^n = x^{m+n} \) 
    \item Division: \(x^m \div x^n = x^{m-n} \)
\end{itemize}
For logarithms:
Logarithms can be thought as the inverse of exponents. The logarithm of \(x^m\) is \(m\log(x)\).
\newline For radicals:
Roots are those are those numbers represented by the radical symbol.
\begin{itemize}
    \item \(x^m \sqrt{x^n} = x^{m/n} \)
    \item \(x^m \sqrt{x^n} = x^{m/n} \)
\end{itemize}

\section{Calculus: Differentiation, integration, and optimization, Chap 5-8}
Upon learning the new concepts of calculus, I found that some visualization would be very helpful. So here are some websites or YouTubers that I think will help you learn the concepts.
\begin{itemize}
    \item  \href{https://www.mathsisfun.com/calculus/index.html}{Quick calculus concepts }
    \item Deeper and longer calc videos explained \href{https://www.3blue1brown.com/topics/calculus}{3blue1brown}
\end{itemize}
with that been said, let's continue to declare the concepts.

\subsection{Derivative}
In a word, derivative is the instantaneous rate of change of a function. In life, we may wanna know how far you can make by walking in 15 minutes, and that's the derivation of distance with respect to time. But time is just one example, we can also ask for the derivation with respect to space, population or anything. And that's the essence of the derivation.

\subsection{Integral}
If the derivative is about the instantaneous rate of change then anti-derivative or the integral is about the sum. The definite integral is a value whereas the indefinite integral is the function. And that's the fundamental truth behind the calculus: change and sum.

\section{Probability, Chap 9-11}
\subsection{Classic probability}
Probability is the formal language to understand how likely a thing would happen in the some circumstances. When learning probability, I found Khan Academy is helpful to understand the concepts: \hyperlink{https://www.khanacademy.org/math/statistics-probability/probability-library}{Khan probability}
\newline To start with a simple case, the probability of flipping a coin with its head on is 1/2. And flipping a coin twice with 2 heads is a joint probability, and flipping a coin with different shape ( assume that will affect the outcome ) is called conditional probability.
\subsection{Discrete distribution}
OK then, you may ask, what if we want to compare the probability of flipping a coin with different shapes or weights? Then we may use a frequency distribution to record how many times heads on for a fixed number of flips with different coins. And that's where the generalization comes.
\newline In a more complicated case, where we have more then just 2 choices, we may wonder what's the probability of drawing one choice from the sample spaces. \newline  Then we need \textbf{Probability Mass Function,(PMF)} it tells us the probability of drawing a specific discrete value from the sample space. You can also think this as a measure of different weights of choices, that's where the mass comes from. 
\newline If the variables are not discrete but continuous, then we need the \textbf{Probability Density Function, (PDF)}
\newline And if we are not asking for a probability with respect to a specific variable but a range that covers some of them, then we call it \textbf{Cumulative Distribution Function, (CDF) } and is defined for both discrete and continuous functions.

\section{Linear Algebra, Chap 12-13}
Upon learning linear algebra, I think this intuitive guide will lead you the way: \hyperlink{https://betterexplained.com/articles/linear-algebra-guide/}{linear algebra guide}
\newline Linear Algebra provides us with a new way to solve the equations and the insight to study multidimentions.
\end{document}
